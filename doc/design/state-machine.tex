
\section{State Machine Code}

Both the PVFS client and server use a state machine model to control
execution.  State machines are written in files with a .sm extention and
are compiled into C data structures for further compilation and linking.

A state machine consists of a set of states, with one state indicated as
the first state.  Each state includes a state action followed by a set of
the possible transitions to other states.  Each transtition includes the
name of the next state and the return code that is used to select that
transition.

Alternatively a state can specify another "nested" state machine rather
than a state action, in which case this new state machine is executed as
a subroutine.  Upon return, the nested state machine returns to the
calling state machine.  A state can also specify that multiple instances
of a state machine will be executed concurrently.  When all of the
concurrent state machines have returned the caller continues.

Transitions can specify another state in the same machine, or to return
from a nested state machine, or to terminate the current state machine.

The following is a synopsis of the state machine language, showing the
various options by way of example:

\begin{verbatim}
/* beginning of .sm file */

/* code at the top of the file is plain C code. */
/* state actions must be declared here before the state machine */

static PINT_sm_action state_action_1 (
    struct PINT_smcb *smcb, job_status_s *js_p);

/* helper functions and other declarations go here too */

%%

/* after the double percent goes the machine declaration */

machine my_machine_sm (
    state_1,
	 state_2,
	 state_3)
{
    state state_1
	 {
	     run state_action_1;
		  default => state_2;
	 }

	 state state_2
	 {
	     jump a_nested_state_machine_sm;
		  default => 3;
	 }

	 state state_3
	 {
	     pjmp state_action_3
		  {
		      4 => parallel_state_machine_1;
		      3 => parallel_state_machine_2;
		  }
		  default => 4;
	 }

	 state state_4
	 {
	     run state_action_4;
		  default => terminate;
	 }
}

%%

/* after the second double percent all code is in plain C */
/* here we implement all of the state actions */

static PINT_sm_action state_action_1 (
    struct PINT_smcb *smcb, job_status_s *js_p)
{
}

static PINT_sm_action state_action_3 (
    struct PINT_smcb *smcb, job_status_s *js_p)
{
}

static PINT_sm_action state_action_4 (
    struct PINT_smcb *smcb, job_status_s *js_p)
{
}

\end{verbatim}


