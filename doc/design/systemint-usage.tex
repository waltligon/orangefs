%
%
\documentclass[11pt, letterpaper]{article}
\usepackage[dvips]{graphicx}
\usepackage{psfig}
\usepackage{rotating}
\usepackage{float}

\graphicspath{{./}{figs/}} 

\restylefloat{table}

\pagestyle{plain}

%
% GET THE MARGINS RIGHT, THE UGLY WAY
%
\topmargin 0.0in
\textwidth 6.5in
\textheight 9.0in
\columnsep 0.25in
\oddsidemargin 0.0in
\evensidemargin 0.0in
\headsep 0.0in
\headheight 0.0in

\title{The PVFS2 System Interface Usage Document}
\author{ PVFS Development Team }
\date{ December, 2002 }

%
% BEGINNING OF DOCUMENT
%
\begin{document}
\maketitle
\tableofcontents
\newpage

\section{Introduction}

This document covers usage notes for the PVFS2 System Interface.  This 
document intends to specify the semantics associated with the System
Interface API.  The System Interface is a client side interface made up
of functions that interact with the PVFS file system. It is intended to
be the low level client side interface that components such as the PVFS2
library and VFS interface are built on top of.

This document will discuss the list of functions provided by the system 
interface, the parameters that need to be passed and the return values
of the functions.  There will also be a mention of the terminology
used in the System Interface the user needs to be aware of.  Each function
description will include an overview of the objective, the function 
parameters, and the return value.

In order to understand the PVFS system interface, we will also need to
discuss a few other concepts and components.  The first such concept is
the PVFS user object.   PVFS user objects are entities that are used to
represent the file system organization and data at the user level i.e.
above the System Interface. These objects comprise files, directories,
symbolic links, and collections.  

\section{User objects}

There are four objects visible to the user and the System Interface supports
various operations on these objects. They are files, directories, symbolic
links, and collections. This document will give an overview of the operations. All PVFS objects have
attributes associated with them.  Some of these are common to all objects,
while others are specific to a particular file system object. As an example,
the distribution parameters would be specific to files. As of now,
operations on symbolic links are not implemented.

It's important to note that attribute operations are by definition
atomic in PVFS, while this is not necessarily true for I/O operations.

%
% FILES
%
\subsection{File}

A file is a logical entity that contains data. PVFS supports operations
to read/write the data in the file and also read/modify the attributes of
the file. The System Interface provides users the ability to distribute
file data over multiple I/O nodes. The file distribution is specified
during its creation.

\subsection{File attributes}
The file attributes that the user could read/modify are:

\begin{itemize}
\item owner
\item group
\item permissions
\item access, modification, and creation times
\item size
\item distribution information.\emph{Can this be modified?}
\end{itemize}

\subsection{Operations on files}

The various operations defined on the file are creation/removal,
attribute modification, and I/O operations such as read/write.

%
% DIRECTORIES
%
\subsection{Directories}

The file system paradigm relies on directories as a mechanism for
organizing objects into a hierarchy. The directory is typically a file
that contains directory entries.

\subsubsection{Directory Attributes}

Directories have no additional attributes defined.

\subsubsection{Operations on directories}

The operations defined on a directory are creation, removal, and
operations on the directory attributes.

%
% SYMBOLIC LINKS
%
\subsection{Symbolic Link}

They are special files that act as pointers or shortcuts to other files.
They do not contain any data.

\subsubsection{Symlink Attributes}

A Symlink has an additional attribute, its {\em target}.  This is a
path, presumably to another PVFS object.  This object does not in fact
have to exist.

\subsubsection{Operations on symlinks}

The operations defined on a symlink are creation, removal and reading
the contents of the link.

%
%COLLECTIONS
%
\subsection{Collection}

A collection is an abstraction for a file system or a group of file
systems. The collection encompasses all the file system objects. The
collection is uniquely identified by a collection identifier.

\subsubsection{Attributes}

A symlink has an additional attribute, its {\em target}.  This is a
path, presumably to another PVFS object.  This object does not in fact
have to exist.


\section{System Interface Concepts}

In order to make the discussion of the System Interface clearer, we will
first describe a few terms that occur frequently in the discussion.

\subsection{Pinode}

The purpose of the pinode is to provide a mechanism for associating
information with a handle.  It is somewhat analogous to an inode in the
Linux kernel.  The pinode structure is maintained only at the system
interface layer and will be transparent to the user.  

\subsection{Pinode Reference}

A pinode reference is an opaque type that is a unique reference to a
PVFS file system object. It is to be noted that the file system objects
are not the same as the objects visible at the user level. All
references to a user object in the System Interface API are in terms of
the object name or the pinode reference.

\section{System interface}

This section provides a list of all the functions defined in the 
system interface. The System Interface API can be logically organized
into 5 groups.

\begin{itemize}
\item\textbf{Interface Management Operations}\\
The interface management operations are initialize and finalize.

\item\textbf{Object creation, query, and destruction operations}\\
The object creation, query, and destruction operations are 
\begin{itemize}
\item lookup
\item getattr
\item setattr
\item mkdir
\item create
\item remove
\item rename
\item symlink
\item readlink
\end{itemize}

\item\textbf{I/O operations}\\
The I/O operations are
\begin{itemize}
\item read
\item write
\item allocate
\item duplicate
\end{itemize}

\item\textbf{Object locking operations}\\
The object locking operations are lock and unlock.

\item\textbf{File system query operations}\\
The file system query operations are statfs, readdir, and fhdump.
\end{itemize}

\section{System Interface function specification}

Most of the System Interface functions have a request and response
structure as parameters. Some functions don't have a response structure
as they don't return a response. This section defines the purpose of
each function along with listing its parameter structures and return
values.

\subsection{Interface Management operations}

\subsubsection{PVFS\_sys\_initialize(pvfs\_mntlist mntent\_list)}

Initialize the System Interface. This function is to be called before
any other System Interface function. The input to the function is a
structure containing configuration information obtained either from the
pvfstab file or the mount command line. The pvfs\_mntlist structure is
shown in table \ref{tab:reqinit}. The pvfs\_mntlist structure contains
a count of the number of mount entry structures and a pointer to the mount
entry structures(refer table. \ref{tab:mntent}).

\begin{table}[H]
\begin{tabular}{|l|l|l|}
\hline 
Field & Type & Description \\
\hline
\hline
nr\_entry & int & number of mounted entries \\
\hline
ptab\_p & pvfs\_mntent* & array pointing to mounted entry details \\
\hline
mt\_lock & gen\_mutex\_t* & mutex lock \\
\hline
\end{tabular}
\caption{struct pvfs\_mntlist}\label{tab:reqinit}
\end{table}

\begin{table}[H]
\begin{tabular}{|l|l|l|}
\hline
Field & Type & Description \\
\hline
\hline
meta\_addr & PVFS\_string & metaserver address \\
\hline
serv\_mnt\_dir & PVFS\_string & root mount point \\
\hline
local\_mnt\_dir & PVFS\_string & local mount point \\
\hline
fs\_type & PVFS\_string & file system type \\
\hline
opt1 & PVFS\_string & options \\
\hline
opt2 & PVFS\_string & options \\
\hline
\end{tabular}
\caption{struct pvfs\_mntent}\label{tab:mntent}
\end{table}

The function returns 0 on success and the negative of the error number
on failure.	

\subsubsection{PVFS\_sys\_finalize(void)}

Shut down the System Interface. This is to be called after all 
System Interface operations are done.

The function returns 0 on success and the negative of the error number
on failure.	

\subsection{Object creation, query, and destruction operations}

\subsubsection{PVFS\_sys\_lookup(
PVFS\_sysreq\_lookup *req,
PVFS\_sysresp\_lookup *resp
)}

Obtain the pinode reference for a file, directory, or a symlink.
This function is used to convert an absolute pathname to a unique
identifier recognised by the file system. The function also performs
permission checking for the entire path that is looked up. The
contents of the request and response structures are shown in
table. \ref{tab:reqlook} and table. \ref{tab:resplook} respectively.

\begin{table}[H]
\begin{tabular}{|l|l|l|}
\hline
Field & Type & Description \\
\hline
\hline
name & PVFS\_string & object name \\
\hline
fs\_id & PVFS\_fs\_id & file system identifier \\
\hline
credentials & PVFS\_credentials & uid, gid, permissions \\
\hline
\end{tabular}
\caption{struct pvfs\_sysreq\_lookup}\label{tab:reqlook}
\end{table}

\begin{table}[H]
\begin{tabular}{|l|l|l|}
\hline
Field & Type & Description \\
\hline
\hline
uid & PVFS\_uid & owner \\ 
\hline
gid & PVFS\_gid & group \\
\hline
perms & PVFS\_permissions & permissions \\
\hline
\end{tabular}
\caption{struct PVFS\_credentials}\label{tab:cred}
\end{table}

\begin{table}[H]
\begin{tabular}{|l|l|l|}
\hline
Field & Type & Description \\
\hline
\hline
pinode\_refn & pinode\_reference & handle, file system id \\
\hline
\end{tabular}
\caption{struct pvfs\_sysresp\_lookup}\label{tab:resplook}
\end{table}

The function returns 0 on success and the negative of the error number
on failure.

\subsubsection{PVFS\_sys\_getattr(
PVFS\_sysreq\_getattr *req,
PVFS\_sysresp\_getattr *resp
)}

Obtain the attributes of a file, directory, or symlink identified by
the pinode reference passed as input. The attributes that could be read
are listed in table. \ref{tab:objattr}. The attribute mask allows the 
option of fetching attributes selectively. The attribute mask is
constructed by oring together the attribute options shown in table.
\ref{tab:optattr}. The request and response structures to the function
are shown in table. \ref{tab:reqgattr} and table. \ref{tab:respgattr}
respectively.

\begin{table}[H]
\begin{tabular}{|l|l|l|}
\hline
Field & Type & Description \\
\hline
\hline
pinode\_refn & pinode\_reference & handle, file system id \\
\hline
attrmask & PVFS\_bitfield & attributes to be fetched \\
\hline
credentials & PVFS\_credentials & uid, gid, permissions \\
\hline
\end{tabular}
\caption{struct pvfs\_sysreq\_getattr}\label{tab:reqgattr}
\end{table}

\begin{table}[H]
\begin{tabular}{|l|l|l|}
\hline
Field & Type & Description \\
\hline
\hline
attr & PVFS\_object\_attr & attributes obtained \\
\hline
extended & PVFS\_attr\_extended & extended attributes \\ 
\hline
\end{tabular}
\caption{struct pvfs\_sysresp\_getattr}\label{tab:respgattr}
\end{table}
	
\begin{table}[H]
\begin{tabular}{|l|l|l|}
\hline
Field & Type & Description \\
\hline
\hline
owner & PVFS\_uid & owner identifier \\
\hline
group & PVFS\_gid & group identifier \\
\hline
perms & PVFS\_permissions & permissions \\
\hline
atime & PVFS\_atime & access time \\ 
\hline
mtime & PVFS\_mtime & modification time \\
\hline
ctime & PVFS\_ctime & creation time \\
\hline
objtype & int & type of file system object \\
\hline
u & union & object specific attributes \\
\hline
\end{tabular}
\caption{struct PVFS\_object\_attr}\label{tab:objattr}
\end{table}

\begin{table}[H]
\begin{tabular}{|l|l|l|}
\hline
Mask & Description \\
\hline
\hline
ATTR\_BASIC & Attributes common to all objects except size \\
\hline
ATTR\_SIZE & File Size \\
\hline
ATTR\_META & Attributes specific to a metafile \\
\hline
ATTR\_DATA & Attributes specific to a datafile \\
\hline
ATTR\_DIR & Attributes specific to a directory \\
\hline
ATTR\_SYM & Attributes specific to a symbolic link \\
\hline
\end{tabular}
\caption{Options for the attribute mask}\label{tab:optattr}
\end{table}


The structures shown in table. \ref{tab:metaattr}, table. \ref{tab:dataattr},
table. \ref{tab:dirattr}, and table. \ref{tab:symattr} make up the union in
table. \ref{tab:objattr}.

\begin{table}[H]
\begin{tabular}{|l|l|l|}
\hline
Field & Type & Description \\
\hline
\hline
dist & PVFS\_Distribution & distribution parameters for file \\
\hline
dfh & PVFS\_Handle* & array of datafile handles \\
\hline
nr\_datafiles & PVFS\_count32 & number of datafiles \\
\hline
\end{tabular}
\caption{PVFS\_metafile\_attr}\label{tab:metaattr}
\end{table}

\begin{table}[H]
\begin{tabular}{|l|l|l|}
\hline
Field & Type & Description \\
\hline
\hline
size & PVFS\_size & size of datafile \\
\hline
dfh & PVFS\_Handle & datafile handle \\
\hline
\end{tabular}
\caption{PVFS\_datafile\_attr}\label{tab:dataattr}
\end{table}

\begin{table}[H]
\begin{tabular}{|l|l|l|}
\hline
Field & Type & Description \\
\hline
\hline
Not Defined &  & \\
\hline
\end{tabular}
\caption{PVFS\_directory\_attr}\label{tab:dirattr}
\end{table}

\begin{table}[H]
\begin{tabular}{|l|l|l|}
\hline
Field & Type & Description \\
\hline
\hline
Not Defined &  & \\
\hline
\end{tabular}
\caption{PVFS\_symlink\_attr}\label{tab:symattr}
\end{table}

The function returns 0 on success and the negative of the error number
on failure.

\subsubsection{PVFS\_sys\_setattr(
PVFS\_sysreq\_setattr *req
)}

Set the attributes of a file, directory, or symlink identified by
the pinode reference passed as input. The attributes that could be set
are listed in table. \ref{tab:objattr}. The attribute mask allows the
option of setting attributes selectively. The attribute mask is
constructed by oring together the attribute options shown in table.
\ref{tab:optattr}. The request structure to the function is shown in
table. \ref{tab:reqsattr}. No response structure is returned to the 
caller function.

\begin{table}[H]
\begin{tabular}{|l|l|l|}
\hline
Field & Type & Description \\
\hline
\hline
pinode\_refn & pinode\_reference & handle, file system id \\
\hline
attr & PVFS\_object\_attr & modified attribute values \\
\hline
attrmask & PVFS\_bitfield & attributes to be set  \\
\hline
credentials & PVFS\_credentials & uid, gid, permissions \\
\hline
extended & PVFS\_attr\_extended & extended attributes \\
\hline
\end{tabular}
\caption{struct pvfs\_sysreq\_setattr}\label{tab:reqsattr}
\end{table}

The function returns 0 on success and the negative of the error
number on failure.

\subsubsection{PVFS\_sys\_mkdir(
PVFS\_sysreq\_mkdir *req,
PVFS\_sysresp\_mkdir *resp
)}

Create a new directory with specified attributes and return a pinode
reference to the created directory. The attribute structure and the mask
specify which attributes are filled in(refer table. \ref{tab:objattr} for
the definition of the attribute structure). The request and response
structures are shown in table. \ref{tab:reqmkdir} and table.
\ref{tab:respmkdir} respectively.

\begin{table}[H]
\begin{tabular}{|l|l|l|}
\hline
Field & Type & Description \\
\hline
\hline
entry\_name & PVFS\_string &  name of new directory \\
\hline
parent\_refn & pinode\_reference & pinode reference of parent directory \\ 
\hline
attr & PVFS\_object\_attr & attributes of new directory \\
\hline
attrmask & PVFS\_bitfield & mask specifying selective attributes \\
\hline
credentials & PVFS\_credentials & uid, gid, permissions \\
\hline
\end{tabular}
\caption{struct pvfs\_sysreq\_mkdir}\label{tab:reqmkdir}
\end{table}

\begin{table}[H]
\begin{tabular}{|l|l|l|}
\hline
Field & Type & Description \\
\hline
\hline
pinode\_refn & pinode\_reference & pinode reference of created directory
\\
\hline
\end{tabular}
\caption{struct pvfs\_sysresp\_mkdir}\label{tab:respmkdir}
\end{table}

The function returns 0 on success and the negative of the error number
on failure.

\subsubsection{PVFS\_sys\_create(
PVFS\_sysreq\_create *req,
PVFS\_sysreq\_create *resp
)}
		
Create a file with specified attributes and return the pinode
reference of the newly created file. The attribute structure is
shown in table. \ref{tab:objattr}. The request and response structures
are shown in table. \ref{tab:reqcreate} and table. \ref{tab:respcreate}
respectively.

\begin{table}[H]
\begin{tabular}{|l|l|l|}
\hline
Field & Type & Description \\
\hline
\hline
entry\_name & PVFS\_string & name of file to be created \\
\hline
parent\_refn & pinode\_reference & pinode reference of parent directory
\\
\hline
attr & PVFS\_object\_attr & attributes of new file \\
\hline
attrmask & PVFS\_bitfield & mask to select attributes \\
\hline
credentials & PVFS\_credentials & uid, gid, permissions \\
\hline
\end{tabular}
\caption{struct pvfs\_sysreq\_create}\label{tab:reqcreate}
\end{table}

\begin{table}[H]
\begin{tabular}{|l|l|l|}
\hline
Field & Type & Description \\
\hline
\hline
pinode\_refn & pinode\_reference & handle, file system id of new file \\
\hline
\end{tabular}
\caption{struct pvfs\_sysreq\_create}\label{tab:respcreate}
\end{table}

The function returns 0 on success and the negative of the error
number on failure.

\subsubsection{PVFS\_sys\_remove(
PVFS\_sysreq\_remove *req
)}

Remove the specified file. The request structure is shown in table.
 \ref{tab:reqrem}. No response structure is returned to the caller
function.

\begin{table}[H]
\begin{tabular}{|l|l|l|}
\hline
Field & Type & Description \\
\hline
\hline
entry\_name & PVFS\_string & name of file to be removed \\
\hline
parent\_refn & pinode\_reference & pinode reference of parent directory
\\
credentials & PVFS\_credentials & uid, gid, permissions \\
\hline
\end{tabular}
\caption{struct pvfs\_sysreq\_remove}\label{tab:reqrem}
\end{table}

The function returns 0 on success and the negative of the error
number on failure.

\subsubsection{PVFS\_sys\_rename(
PVFS\_sysreq\_rename *req
)}

Rename the specified file. The request structure is shown in table.
 \ref{tab:reqrename}. No response structure is returned to the caller
function.

\begin{table}[H]
\begin{tabular}{|l|l|l|}
\hline
Field & Type & Description \\
\hline
\hline
old\_entry & PVFS\_string & old file name \\
\hline
old\_parent\_reference & pinode\_reference & pinode reference of old parent \\
\hline
new\_entry & PVFS\_string & new file name \\
\hline
pinode\_reference & new\_parent\_reference & pinode reference of new
parent  \\
\hline
fs\_id & PVFS\_fs\_id & file system id \\
\hline
credentials & PVFS\_credentials & uid, gid, permissions \\
\hline
\end{tabular}
\caption{struct pvfs\_sysreq\_rename}\label{tab:reqrename}
\end{table}

The function returns 0 on success and the negative of the error
number on failure.

\subsubsection{PVFS\_sys\_symlink(
PVFS\_sysreq\_symlink *req,
PVFS\_sysresp\_symlink *resp
)}

Create a symbolic link to specified file with given attributes. The
request structure is shown in table. \ref{tab:reqsym} and the response
structure is shown in table. \ref{tab:respsym} respectively. The
attribute structure is shown in table. \ref{tab:objattr}.

\begin{table}[H]
\begin{tabular}{|l|l|l|}
\hline
Field & Type & Description \\
\hline
\hline
name & PVFS\_string & name of symbolic link \\
\hline
fs\_id & PVFS\_fs\_id & file system id \\
\hline
target & PVFS\_string & file to create a link to \\
\hline
attr & PVFS\_object\_attr & attributes of link \\
\hline
attrmask & PVFS\_bitfield & mask to select attributes \\
\hline
credentials & PVFS\_credentials & uid, gid, permissions \\
\hline
\end{tabular}
\caption{struct pvfs\_sysreq\_symlink}\label{tab:reqsym}
\end{table}

\begin{table}[H]
\begin{tabular}{|l|l|l|}
\hline
Field & Type & Description \\
\hline
\hline
pinode\_refn & pinode\_reference & pinode reference of symbolic link \\ 
\hline
\end{tabular}
\caption{struct pvfs\_sysresp\_symlink}\label{tab:respsym}
\end{table}

The function returns 0 on success and the negative of the error
number on failure.

\subsubsection{PVFS\_sys\_readlink(
PVFS\_sysreq\_readlink *req,
PVFS\_sysresp\_readlink *resp
)}

Return the file name pointed to by the specified symbolic link. The request
structure is shown in table. \ref{tab:reqreadlk} and the response structure
is shown in table. \ref{tab:respreadlk} respectively. 

\begin{table}[H]
\begin{tabular}{|l|l|l|}
\hline
Field & Type & Description \\
\hline
\hline
pinode\_refn & pinode\_reference & pinode reference of symbolic link \\
\hline
credentials & PVFS\_credentials & uid, gid, permissions \\
\hline
\end{tabular}
\caption{struct pvfs\_sysreq\_readlink}\label{tab:reqreadlk}
\end{table}

\begin{table}[H]
\begin{tabular}{|l|l|l|}
\hline
Field & Type & Description \\
\hline
\hline
target & PVFS\_string & target pointed to by symbolic link \\
\hline
\end{tabular}
\caption{struct pvfs\_sysresp\_readlink}\label{tab:respreadlk}
\end{table}

The function returns 0 on success and the negative of the error
number on failure.

\subsection{Input/Output operations}

\subsubsection{PVFS\_sys\_read(
PVFS\_sysreq\_read *req,
PVFS\_sysresp\_read *resp
)}

Read data from a given file specified by the pinode reference
and the I/O description. The request and response structures
are not yet specified.

\begin{table}[H]
\begin{tabular}{|l|l|l|}
\hline
Field & Type & Description \\
\hline
\hline
Not defined & & \\
\hline
\end{tabular}
\caption{struct pvfs\_sysreq\_read}\label{tab:reqread}
\end{table}

The function returns 0 on success and the negative of the error
number on failure.

\subsubsection{PVFS\_sys\_write(
PVFS\_sysreq\_write *req,
PVFS\_sysresp\_write *resp
)}

Write data to a file specified by the pinode reference and the
I/O description. The request and reponse structures are not yet 
specified.

\begin{table}[H]
\begin{tabular}{|l|l|l|}
\hline
Field & Type & Description \\
\hline
\hline
Not defined & & \\
\hline
\end{tabular}
\caption{struct pvfs\_sysreq\_write}\label{tab:reqwrite}
\end{table}

The function returns 0 on success and the negative of the error
number on failure.

\subsubsection{PVFS\_sys\_allocate(
PVFS\_sysreq\_allocate *req,
PVFS\_sysresp\_allocate *resp
)}

Allocate space of specified size on I/O servers indicated in the
request. The request and response structures are not yet specified.

\begin{table}[H]
\begin{tabular}{|l|l|l|}
\hline
Field & Type & Description \\
\hline
\hline
Not defined & & \\
\hline
\end{tabular}
\caption{struct pvfs\_sysreq\_allocate}\label{tab:reqalloc}
\end{table}

The function returns 0 on success and the negative of the error
number on failure.

\subsubsection{PVFS\_sys\_duplicate(
PVFS\_sysreq\_duplicate *req,
PVFS\_sysresp\_duplicate *resp
)}

Create a new file that has the same distribution and attributes as
specified in the request. The file to be duplicated is indicated by its
pinode reference and the new file is indicated by its name and the
pinode reference of the directory that contains it. The request and
response structures are shown in table. \ref{tab:reqdup} and table.
\ref{tab:respdup} respectively.

\begin{table}[H]
\begin{tabular}{|l|l|l|}
\hline
Field & Type & Description \\
\hline
\hline
old\_reference & pinode\_reference & pinode reference of file to be
duplicated \\
\hline
new\_entry & PVFS\_string & name of new file \\
\hline
new\_parent\_reference & pinode\_reference & pinode reference of the
parent of new file \\
\hline
fs\_id & PVFS\_fs\_id & file system identifier \\
\hline
\end{tabular}
\caption{struct pvfs\_sysreq\_duplicate}\label{tab:reqdup}
\end{table}

\begin{table}[H]
\begin{tabular}{|l|l|l|}
\hline
Field & Type & Description \\
\hline
\hline
pinode\_refn & pinode\_reference & handle, fs id \\
\hline
\end{tabular}
\caption{struct pvfs\_sysresp\_duplicate}\label{tab:respdup}
\end{table}

The function returns 0 on success and the negative of the error
number on failure.

\subsection{Object locking operations}

\subsubsection{PVFS\_sys\_lock(
PVFS\_sysreq\_lock *req,
PVFS\_sysresp\_lock *resp
)}

Not defined.

\begin{table}[H]
\begin{tabular}{|l|l|l|}
\hline
Field & Type & Description \\
\hline
\hline
Not defined & & \\
\hline
\end{tabular}
\caption{struct pvfs\_sysreq\_lock}\label{tab:reqlock}
\end{table}

The function returns 0 on success and the negative of the error
number on failure.

\subsubsection{PVFS\_sys\_unlock(
PVFS\_sysreq\_unlock *req,
PVFS\_sysresp\_unlock *resp
)}

Not defined.

\begin{table}[H]
\begin{tabular}{|l|l|l|}
\hline
Field & Type & Description \\
\hline
\hline
Not defined & & \\
\hline
\end{tabular}
\caption{struct pvfs\_sysreq\_unlock}\label{tab:requnlock}
\end{table}

The function returns 0 on success and the negative of the error
number on failure.

\subsection{File system query operations}

\subsubsection{PVFS\_sys\_statfs(
PVFS\_sysreq\_statfs *req,
PVFS\_sysresp\_statfs *resp
)}

Statfs returns the statistics of the file system specified by the id in
the request. The returned information is separated into metaserver related
and I/O server related information. The request structure is shown in
table. \ref{tab:reqstatfs} and the response structure is shown in table.
\ref{tab:respstatfs} respectively. The response structure contains the
PVFS\_statfs structure shown in table. \ref{tab:statfs}. The PVFS\_statfs
structure is divided into metaserver and I/O server statistics. The
metaserver statistics structure is displayed in table. \ref{tab:metastat}
and the I/O server statistics structure is shown in table. \ref{tab:iostat}.

\begin{table}[H]
\begin{tabular}{|l|l|l|}
\hline
Field & Type & Description \\
\hline
\hline
fs\_id & PVFS\_fs\_id & file system identifier \\
\hline
credentials & PVFS\_credentials & uid, gid, permissions \\
\hline
\end{tabular}
\caption{struct pvfs\_sysreq\_statfs}\label{tab:reqstatfs}
\end{table}

\begin{table}[H]
\begin{tabular}{|l|l|l|}
\hline
Field & Type & Description \\
\hline
\hline
statfs & PVFS\_statfs & statistics - metaserver and I/O server\\
\hline
\end{tabular}
\caption{struct pvfs\_sysresp\_statfs}\label{tab:respstatfs}
\end{table}

\begin{table}[H]
\begin{tabular}{|l|l|l|}
\hline
Field & Type & Description \\
\hline
\hline
mstat & PVFS\_meta\_stat & statistics - metaserver \\
\hline
iostat & PVFS\_io\_stat & statistics - I/O server \\
\hline
\end{tabular}
\caption{PVFS\_statfs}\label{tab:statfs}
\end{table}

\begin{table}[H]
\begin{tabular}{|l|l|l|}
\hline
Field & Type & Description \\
\hline
\hline
filetotal & PVFS\_count32 & total number of metafiles on server \\
\hline
\end{tabular}
\caption{PVFS\_meta\_stat}\label{tab:metastat}
\end{table}

\begin{table}[H]
\begin{tabular}{|l|l|l|}
\hline
Field & Type & Description \\
\hline
\hline
blksize & PVFS\_size & file system block size \\
\hline
blkfree & PVFS\_count32 & number of file blocks \\
\hline
blktotal & PVFS\_count32 & total number of blocks available \\
\hline 
filetotal & PVFS\_count32 & total number of files \\
\hline 
filefree & PVFS\_count32 & number of free files \\
\hline
\end{tabular}
\caption{PVFS\_io\_stat}\label{tab:iostat}
\end{table}

The function returns 0 on success and the negative of the error
number on failure.

\subsubsection{PVFS\_sys\_fhdump(
PVFS\_sysreq\_fhdump *req,
PVFS\_sysresp\_fhdump *resp
)}

Not defined.

\begin{table}[H]
\begin{tabular}{|l|l|l|}
\hline
Field & Type & Description \\
\hline
\hline
Not defined & & \\
\hline
\end{tabular}
\caption{struct pvfs\_sysreq\_fhdump}\label{tab:reqfhdump}
\end{table}

The function returns 0 on success and the negative of the error
number on failure.

\subsubsection{PVFS\_sys\_readdir(
PVFS\_sysreq\_readdir *req,
PVFS\_sysresp\_readdir *resp
)}

Readdir is used to read the contents of a directory. The directory
to be read is specified by the pinode reference of the directory
and number of entries to be read is indicated by the incount in the
request. The token field is an opaque type that is initialized to
TOKEN\_START. If the specified directory entries can't be read then
the outcount in the response has the number of entries actually read.
The readdir request can be repeatedly used to read a fixed number of
directory entries each time until all the entries in the directory
are covered. In this case, the token returned in each response is
passed in as the token in the subsequent request. This is done until
all entries are read. The request and response structures are shown 
in table. \ref{tab:reqreaddir} and table. \ref{tab:respreaddir} 
respectively. The directory entry is shown in table. \ref{tab:dirent}.

\begin{table}[H]
\begin{tabular}{|l|l|l|}
\hline
Field & Type & Description \\
\hline
\hline
pinode\_refn & pinode\_reference & pinode reference of directory to be
read \\
\hline
token & PVFS\_token & opaque type indicating current position in directory \\
\hline
pvfs\_dirent\_incount & PVFS\_count32 & number of directory entries to
be read \\
\hline
\hline
\end{tabular}
\caption{struct pvfs\_sysreq\_readdir}\label{tab:reqreaddir}
\end{table}

\begin{table}[H]
\begin{tabular}{|l|l|l|}
\hline
Field & Type & Description \\
\hline
\hline
token &  PVFS\_token &  current position in directory \\
\hline
pvfs\_dirent\_outcount & number of entries actually read \\
\hline
dirent\_array & PVFS\_dirent* & array of directory entries read \\
\hline
\end{tabular}
\caption{struct pvfs\_sysresp\_readdir}\label{tab:respreaddir}
\end{table}

\begin{table}[H]
\begin{tabular}{|l|l|l|}
\hline
Field & Type & Description \\
\hline
\hline
d\_name[PVFS\_NAME\_MAX + 1] & char & directory entry name \\
\hline
\end{tabular}
\caption{PVFS\_dirent}\label{tab:dirent}
\end{table}

The function returns 0 on success and the negative of the error
number on failure.

\section{Packaging}

This has to be decided.  How the system interface will be accessed by
the user program.

\end{document}











